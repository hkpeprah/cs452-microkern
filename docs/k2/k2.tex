\documentclass[12pt]{article}
\usepackage{enumerate, amsmath, fullpage, hyperref, amsfonts, titlesec, listings}
\renewcommand*\contentsname{Table of Contents}
\newlength\tindent
\setlength{\tindent}{\parindent}
\setlength{\parindent}{0pt}
\renewcommand{\indent}{\hspace*{\tindent}}
\lstset{
   breaklines=true,
   basicstyle=\scriptsize\rmfamily}
\begin{document}
\thispagestyle{empty}
\begin{center}
  {\bf\Large Kernel 2}\\
  {\bf\large CS452 - Spring 2014}\\
  Real-Time Programming\vspace{5cm}\\
  {\bf Team }\\
  Max Chen - mqchen\\
  mqchen@uwaterloo.ca\\[1\baselineskip]
  Ford Peprah - hkpeprah\\
  ford.peprah@uwaterloo.ca\vspace{5cm}\\
  Bill Cowan\\
  University of Waterloo\\
  {\bf Due Date:} Friday, $30^{th}$, May, $2014$
\end{center}
\newpage
% Program Description: how to operate it, full pathname
% Description fo structure of Kernel: algorithms, data structures, etc.
% Location fo source code + MD5
% Output produced by program and explanation of why it does
\thispagestyle{empty}
\tableofcontents
\newpage
\section{Program Description}
\subsection{Getting the Program}
To run the program, one must have read/write access to the source code, as well as the ability to make and run the program.  Before attempting to run the pogram ensure that the following three conditions are met:
\begin{itemize}
  \item You are currently logged in as one of \texttt{cs452}, \texttt{mqchen}, or \texttt{hkpeprah}.
  \item You have a directory in which to store the source code, \\ e.g. \texttt{\~/cs452\_microkern\_mqchen\_hkpeprah}.
  \item You have a folder on the FTP srever with your username, e.g. \texttt{/u/cs452/tftp/ARM/cs452}.
\end{itemize}
First, you must get a copy of the code.  To to this, log into one of the aforementioned accounts and change directories to the directory you created above (using \texttt{cd}), then run one of
\begin{center}
  \begin{verbatim}
    git clone file:////u8/hkpeprah/cs452-microkern -b kernel2 .
                           or
    git clone file:////u7/mqchen/cs452/cs452-microkern -b kernel2 .
  \end{verbatim}
\end{center}
\vspace{-0.5cm}You will now have a working instance of our kernel2 source code in your current directory.  To make the application and upload it to the FTP server at the location listed above (\texttt{/u/cs452/tftp/ARM/YOUR\_USERNAME}), run \texttt{make upload}.
\\[1\baselineskip]

\subsection{Running the Program}
To run the application, you need to load it into the RedBoot terminal.  Ensure you've followed the steps listed above in the ``Getting the Program'' settings to ensure you have the correct directories and account set up.  Navigate to the directory in which you cloned the source code and run \texttt{make upload}.  The uploaded code should now be located at
\begin{center}
  \texttt{/u/cs452/tftp/ARM/YOUR\_USERNAME/assn2.elf}
\end{center}
To run the application, go to the RedBoot terminal and run the command
\begin{center}
  \texttt{load -b 0x00218000 -h 10.15.167.4 ``ARM/YOUR\_USERNAME/assn2.elf''; go}
\end{center}
The application should now begin by running through the game tasks before reaching a prompt.  The generated files will be located in \texttt{DIR/build} where \texttt{DIR} is the directory you created in the earlier steps.  To access and download an existing version of the code, those can be found at \texttt{/u/cs452/tftp/ARM/mqchen/assn2.elf} and \texttt{/u/cs452/tftp/ARM/hkpeprah/assn2.elf}.
\\[2\baselineskip]

\section{Kernel Structure}
\subsection{System Calls}
\section{Game Tasks}
\subsection{Priorities}
\begin{center}
  \begin{tabular}{|l|l|l|}
    \hline
    Task Name & Task ID & Priority \\\hline
    FirstTask       & $0$  & $15$ \\\hline
    NameServer      & $1$  & $15$ \\\hline
    Server          & $2$  & $11$ \\\hline
    Client (NXCLZ)  & $3$  & $7$ \\\hline
    Client (HIIJS)  & $4$  & $8$ \\\hline
    Client (JUWKR)  & $5$  & $5$ \\\hline
    Client (YEOYF)  & $6$  & $1$ \\\hline
    Client (NLZEM)  & $7$  & $7$ \\\hline
    Client (GXKFQ)  & $8$  & $3$ \\\hline
    Client (LPEKV)  & $9$  & $1$ \\\hline
    Client (CXWTY)  & $10$ & $6$ \\\hline
    Client (ABJFT)  & $11$ & $8$ \\\hline
    Client (TXRHX)  & $12$ & $6$ \\\hline
  \end{tabular}
  \\
\end{center}
The priority for the FirstTask (the task that bootraps the NameServer, Server, and Clients/Players) was chosen to be $15$ because $15$ is the highest possible priority supported by our kernel, thus allowing the task to proceed without yielding to another task, self the NameServer which blocks on \texttt{Receive}, to allow it to create all the tasks as soon as possible before exiting.  The priority of the NameServer was chosen to be $15$ to ensure that the NameServer was created and did any of the necessary setup work before any other task that would need it was to be run.  This ensure that any task calling \texttt{RegisterAs} or \texttt{WhoIs} would succeed, as the NameServer \texttt{RCV_BLK}s (Receive Blocks) as the last step in its setup.  So before any task needs the NameServer, it is already setup and waiting to be sent messages.  The priorities of the Server and the Clients are unimportant with respect to one another, as any client with a higher priority than the server would just block waiting for the server to respond to it; the priorities were essentially random to allow for diversity in the result.  The only important factor was to ensure that the priorities were less than the NameServer to ensure that the NameServer was ready before they would need it.  There were two different approaches to do this; have the first task send a message to the nameserver, blocking and allowing the nameserver to setup and respond before creating the remaining tasks, or have the other tasks lower priority so that the NameServer was the next task scheduled after the first task exited; we chose the latter approach.
\\[1\baselineskip]

\subsection{Game Task Output}
The output from the GameTask is as follows:
\begin{verbatim}
Player HIIJS(Task 4) throwing PAPER
Player ABJFT(Task 11) throwing ROCK
HIIJS won the round
Press any key to continue:

Player NXCLZ(Task 3) throwing PAPER
Player NLZEM(Task 7) throwing SCISSORS
NLZEM won the round
Press any key to continue:

Player CXWTY(Task 10) throwing ROCK
Player TXRHX(Task 12) throwing SCISSORS
CXWTY won the round
Press any key to continue:

Player JUWKR(Task 5) throwing SCISSORS
Player GXKFQ(Task 8) throwing SCISSORS
Round was a TIE
Press any key to continue:

Player JUWKR(Task 5) throwing ROCK
Player GXKFQ(Task 8) throwing PAPER
GXKFQ won the round
Press any key to continue:

Player YEOYF(Task 6) throwing SCISSORS
Player LPEKV(Task 9) throwing ROCK
LPEKV won the round
Press any key to continue:
\end{verbatim}
The implementation of \texttt{random} using a set seed, so the results from the game are deterministic, which allows us to argue that the results will always be the same as above.  First, the explanation of how Rock-Papers-Scissors works.  To begin a game of Rock-Paper-Scissors, two parties must agree to play, at which point, each party throws one of \{Rock, Paper, Scissors\} simulataenously.  Rock beats Scissors, Scissors beats Paper, and for some god awful reason, Paper beats Rock.  If both parties throw the same hand, the round ends in a tie, and neither party is victorious.
\\[2\baselineskip]

\section{Performance Measurements}
\subsection{Results}
\begin{tabular}{|c|c|c|c|l|}
  \hline
  {\bf Message Length} & {\bf Caches} & {\bf Send Before Receive*} & {\bf Optimization } & {\bf Microseconds} \\\hline
  $4$ bytes & off & yes & off & $343.8453713$ \\\hline
  $64$ bytes & off & yes & off & $462.8687691$ \\\hline
  $4$ bytes & on & yes & off & $24.41505595$ \\\hline
  $64$ bytes & on & yes & off & $31.53611394$ \\\hline
  $4$ bytes & off & no & off & $378.4333672$ \\\hline
  $64$ bytes & off & no & off & $496.439471$ \\\hline
  $4$ bytes & on & no & off & $27.46693795$ \\\hline
  $64$ bytes & on & no & off & $35.60528993$ \\\hline
  $4$ bytes & off & yes & on & $192.2685656$ \\\hline
  $64$ bytes & off & yes & on & $231.9430315$ \\\hline
  $4$ bytes & on & yes & on & $12.20752798$ \\\hline
  $64$ bytes & on & yes & on & $15.25940997$ \\\hline
  $4$ bytes & off & no & on & $215.6663276$ \\\hline
  $64$ bytes & off & no & on & $255.3407935$ \\\hline
  $4$ bytes & on & no & on & $14.24211597$ \\\hline
  $64$ bytes & on & no & on & $16.27670397$ \\\hline
\end{tabular}
{\tiny {\bf *} - Assignment says ``Send Before Reply'', however, replies are non-blocking and don't depend on a send to occur.}
\\[1\baselineskip]
\subsection{Explanation}
Something something...
\\[2\baselineskip]
\section{MD5 Hashes}
Source files can be accessed at either \texttt{/u7/mqchen/cs452/cs452-microkern} or \\ \texttt{/u8/hkpeprah/cs452-microkern}. The MD5 hashes of the source files are as follows:\\
% \lstinputlisting{md5}
\end{document}
