\documentclass[12pt]{article}
\usepackage{enumerate, amsmath, fullpage, hyperref, amsfonts, titlesec, listings}
\renewcommand*\contentsname{Table of Contents}
\newlength\tindent
\setlength{\tindent}{\parindent}
\setlength{\parindent}{0pt}
\renewcommand{\indent}{\hspace*{\tindent}}
\lstset{
   breaklines=true,
   basicstyle=\scriptsize\rmfamily}
\begin{document}
\thispagestyle{empty}
\begin{center}
  {\bf\Large Kernel 3}\\
  {\bf\large CS452 - Spring 2014}\\
  Real-Time Programming\vspace{5cm}\\
  {\bf Team }\\
  Max Chen - mqchen\\
  mqchen@uwaterloo.ca\\[1\baselineskip]
  Ford Peprah - hkpeprah\\
  ford.peprah@uwaterloo.ca\vspace{5cm}\\
  Bill Cowan\\
  University of Waterloo\\
  {\bf Due Date:} Monday, $9^{th}$, June, $2014$
\end{center}
\newpage
% Program Description: how to operate it, full pathname
% Description fo structure of Kernel: algorithms, data structures, etc.
% Location fo source code + MD5
% Output produced by program and explanation of why it does
\thispagestyle{empty}
\tableofcontents
\newpage
\section{Program Description}
\section{Kernel Structure}
\section{Program Output}
\section{MD5 Hashes}
Source files can be accessed at either \texttt{/u7/mqchen/cs452/cs452-microkern} or \\ \texttt{/u8/hkpeprah/cs452-microkern} on the \texttt{kernel3} branch. The listing of all the fields being submitted alongisde their MD5 hashes and locations are as follows:
\lstinputlisting{md5}
\end{document}
